\subsection{Role of a FPC Center}
%Rationale for Center Approach (one-page) (Explain the unique opportunities that an integrated Center will provide and describe what will be achieved in the Center mode that could not be achieved with group or individual support. Discuss the long-term strategic goals of an integrated Center. Describe the potential legacy and national and global impact of the proposed Center.)
%
%(3.b) Research Objectives (eight to ten pages): 
%Vision and long-range research goals (also describe proposed research areas/themes, how they integrate with each other to realize the Center's research vision)
%
%
%Indicate the potential impact or expected significance the Center's research will have on the Nation's scientific and/or technological base. 
%
%Include a description of current research activities and, if the proposed Center research is closely related to ongoing research at an existing Center (e.g., an STC, ERC, CEMRI/MRSEC, SLC or national laboratory), explain how the research activities of the proposed Center complement as well as differ from those of the existing Center(s). 
%
%Relationship to other regional, national, and international programs

\subsubsection{Goals}
The primary goal of the center is to investigate, build, distribute, and maintain software that will spur innovation in scientific computing over an extended period of time. The center should stay abreast of current and emerging technologies like multi-core, GPU and FPGA accelerators for supporting and rapidly advancing science, and provide guidance to the scientific user community. A strong software design and development team should build and maintain reliable, usable, and extensible software, organically and from user, developer, and business communities. The center should be a knowledge repository that provides tutorials and training on new software and technologies supported by the center, offers consulting services to help scientists use the available software including long-term engagement, and shares best practices that may even be standardized.

\subsubsection{Software Framework and Repository}
A sustainable software infrastructure requires building the ``software right". The center should adopt portable, open, extensible software development frameworks and practices to allow easy intake, sharing and long-term maintenance of code. These promote small businesses and research developers to contribute both open and proprietary software to the center, and build on existing work. Platforms like Eclipse and OpenFabric prevailing in the open source can be reused for this.

A software repository should form the center-piece. Software should be organized in a modular fashion with stakeholders identified for each module to ensure long-term maintenance and use. Center staff with strong software development skills and domain knowledge should screen community contributions to ensure that development practices are met, and the contributions help support the targeted scientific research. Code contribution history and bug tracking features are necessary.

The center should encourage growth of new software while balancing the need to maintain the previous codebase in a manner accessible to scientists and students. Users in specific domains (e.g. genomics, social networks) or of specific technologies (e.g. FPGA, GPU) should be able to download all related software packages to get started with minimal overhead. A framework for visual composition of scientific applications, that allows different domain modules to be pipelined or technology modules to be swapped in/out, is needed. Scientific workflows provide some of these features.

The center should plan for the lifecycle of software. As technology matures and critical mass in a user community is achieved, active development of a software module may transition to one of management, maintenance and even deprecation. Approaches such as packaging hardware accelerators like GPUs, FGPA and multi-core machines along with requisite software platform into appliances can also be useful. This is already happening in next generation sequencing machines. Testing and benchmarking of software is also necessary to ensure verified code with known performance characteristics is present.

Documentation of both software and best practices is essential. Software documentation should assist with both use and further development of the software. The center should identify software,
technology, techniques and algorithms that work best for specific domains and applications. Design patterns and usage guidelines for advanced tools should be available, with the intent of helping users select the right set of software modules for their application and of promoting better development of their applications using advanced technologies. They should provide scientists with sufficient information on the tradeoff between the effort required to incorporate a new software or programming model against the potential gains for their application. Best practices and case studies should be recorded with contributions from center staff, researchers, and industry partners, possibly using a wiki-style system. Viability of new technology for specific scientific applications should be discussed. Knowledge gained and lessons learnt as part of on-boarding new tools should be recorded as they outlive individual techniques. The center should provide an open forum for discussion between the various stakeholders, and help evangelize and encourage community participation and contribution.

\subsubsection{Scoping the Communities}
Choosing the ``right software" to maintain in the center is important. The scope of applications that the center assimilates or builds should be useful in advancing the sciences. The center should target science domains and applications that have impact in the near and long term (2-3 years). The success of the center depends on buy-in from the domain scientists into the software provided. This requires the scientists to be shown the advantages of the new tools and technology. Working closely with a few domain users on software and applications that show immediate results is useful to gain trust, and help start engagement with long-term benefits.

A balanced selection of high risk software applications that can potentially advance sciences radically, as well as software with incremental, sustained improvements should be chosen. Engaging with scientists who are making swift advances in software tools in their domain and helping them build those in the center’s software framework will help push the domain forward. At the same time, working with the broader, underserved community to propagate software tools maintained by the center is necessary. Domains like biology and social networks, which have limited legacy code and many low hanging opportunities, are well suited for a center’s focus.

Both cross-cutting software packages like graph algorithms that can be used across domains, as well as software that integrates vertically for a domain have to be developed. Working with multiple domains can help identify common algorithms and packages in otherwise unrelated fields. Using a uniform software framework and repository will make this transition between communities easier.

The center should provide avenues for computer science investigators to participate to ensure rapid incorporation of advanced software tools. Incentives for such researchers include access to the center’s pool of domain users who can help guide the research direction and increase the adoption and impact of their software.
