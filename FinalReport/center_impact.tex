\subsection{Impact of the Center for Sustainable Software on Future Computing Platforms}
We see the
subsequent center proposal itself as impacting both science and technology. The center will close
the loop between exploratory computing platforms and practical researchers. Scientists will gain
access to advanced, possibly disruptive and ground-breaking technology through a well designed
software infrastructure that minimizes non-science effort. Technology companies and researchers
will gain access to the forefront of sustainable scientific applications. Collaborations between
biologists, social scientists and computer scientists have the potential to improve US economic
competitiveness, national and world health, and retain the US pre-eminence in these vital domains.

A center with an on-line presence provides a nexus of information available across educational
levels and geographical boundaries. Part of the conceptualization process is identifying available
resources and how best to deliver them for educators’ as well as researchers’ use. A center that can
provide software and datasets to simulate and visualize natural phenomena in these sciences will
help advance classroom learning, and engage US citizens in science and policy~\cite{Xie24062011}.
