\subsection{Motivation}
Knowledge has often been forgotten during the course of recorded history.
The historical record is quite clear that important knowledge is often ``rediscovered'' as part of the human experience.
The modern computer era has been a double-edged sword in regard to this subject.
On one hand, data can now be stored for long periods of time and reproduced on demand. 
On the other hand, creators of knowledge beyond the bleeding edge, in particular academia,
will discover and innovate in some area and then move on. 
Being beyond the bleeding edge, the lone researcher (or small research group) finds it hard to provide an infrastructure that can sustain innovation beyond that edge for very long.

Acceleration technology has long suffered from losing cutting-edge research knowledge. 
In the 1970's commercial general purpose computers had very low floating point performance because accounting was the main application in use at that time. 
Universities as well as the military had a need for high performance floating point mathematics to solve vector and matrix calculations. 
As a result, SIMD, MIMD, systolic, and vector machines were conceived and high level languages were designed to program them. 
Many roadblocks were overcome to learn to program these exotic machines. 
Ironically the winning technology was a CISC/RISC single core von-Neumann architecture called the microprocessor. 
In the process of choosing a winner, knowledge from many of the hard won programming battles was lost in the 1980's. 
We must now relearn how to overcome the roadblocks of the past as well as address the current roadblocks created by the high performance computing requirements of present day science. 
We must also engage researchers who recall past successes and failures in technologies like vectorization and network-specific optimal algorithms.


This recurrent problem can be addressed by creating a framework that will allow researchers to express their innovations in a reusable, sustainable fashion, 
by providing a working archive of best practices to minimize the need to relearn these lessons in the future, 
and by providing continuing opportunities for established researchers as well as future generations of scientific software developers.
By creating centers staffed by personnel capable of ``institutionalizing'' the researchers' innovations, 
the academic community can leverage and keep safe and useful the hard earned knowledge that will make our nation stronger.

%One problem has been that visionary research discovers algorithms proven correct but infeasible for existing computational infrastructure.

%%% Local Variables: 
%%% mode: latex
%%% TeX-master: "FPC_FinalReportMain"
%%% End: 
