\subsection{Organization}
The conceptualization process was led by the PIs from University of Southern California,
Georgia Institute of Technology and Rutgers, and guided a steering committee of
leading experts. 
The investigators are renowned experts on future computing platforms applied to
diverse scientific domains and in managing large scale software infrastructure. 
The USC Center for
Advanced Software Technologies, directed by Prasanna, 
and the NSF I/UCRC Center for Hybrid
Multicore Productivity Research, led by Bader, 
provide an available pool of software and technology
resources for the conceptual design. 
The investigators have successful prior collaborations. 
The coordination for the design was achieved though frequent virtual meetings using audio and video
conferencing, shared ``wiki'' sites and document repositories, and discussion lists. 
All materials are open to the steering committee and workshop participants, and will be archived.

\subsubsection{PIs, Co-PIs and Senior Personnel}

\paragraph{Viktor Prasanna} Professor, Electrical Engineering and Computer Science, University of Southern
California (USC), is the Principal Investigator of the proposal. He is an expert on parallel and
reconfigurable computing. He provides expertise on FPGA and multi-core platforms..

\paragraph{David Bader} Professor, Computational Science and Engineering, Georgia Institute of Technology
(GT), is a Co-Principal Investigator and expert on high-performance computing as applied to
computational biology, genomics and massive-scale data analytics. He provides expertise on
GPGPU and many-core platforms..

\paragraph{Manish Parashar} Professor, Electrical and Computer Engineering, Rutgers, is a Co-Principal Investigator.
He is an expert on Cyberinfrastructure. He provides expertise on Cloud computing
and software infrastucture.

\paragraph{Rich Vuduc} Assistant Professor, Computational Science and Engineering, GT, is a Co-Principal
Investigator and expert in high-performance computing on irregular data. He provides
expertise on GPGPU and multi-core platforms.


\paragraph{Yogesh Simmhan} Senior Research Associate, Electrical Engineering, USC, is a Co-Principal
Investigator. He is an expert on Cloud Computing and scientific data management. He
provides expertise on Cloud platforms and software infrastucture..

\paragraph{Jason Riedy} Research Scientist, GT, is a Co-Principal Investigator. His expertise in social networks
and parallel algorithms provides technical capability on multithreaded platforms and software
infrastucture. 

\paragraph{Shel Swenson} Postdoctoral Research Associate Electrical Engineering, USC, Project Manager.
 She provides cross-cutting expertise in algorithm design and computational biology, and
was responsible for facilitating the activities of the conceptual design, conducting literature
reviews, coordinating writing of technical reports, and helping organize the workshops and meetings.

\subsubsection{Steering Committee}

\paragraph{Srinivas Aluru} Professor, Computational Science and Engineering, GT.
Role: Identifying grand challenge problems in bioinformatics and genomics domains,
informing software infrastructure design.

\paragraph{Neil Chue Hong} Director, Software Sustainability Institute, UK. Dr. Chue Hong’s Institute,
funded by the UK EPSRC, represents the software interests of UK researchers nationally
and internationally. He oversees the operations and manages collaborations. Earlier, Dr.
Chue Hong was Director of the OMII-UK open-source e-Research software community and
Technical Manager of the JISC-funded NeISS social simulation project. Role: International
expertise on software infrastructure, managing a scientific software center.

\paragraph{Steven Salzberg} Professor, Departments of Medicine, Biostatistics, and Computer Science
Director, Center for Computational Biology
McKusick-Nathans Institute of Genetic Medicine
Johns Hopkins University School of Medicine.
Role: Characterizing grand challenge problems in the biomedical and genomics domains.

\paragraph{Rob Schreiber} Assistant Director of the Exascale Computing Lab, Hewlett Packard Labs.

\paragraph{Marc Snir} Professor, Department of Computer Science, University of Illinois at Urbana-Champaign and  Director, Department of Mathematics and Computer Science, Argonne National Labs.

\paragraph{Sanjiva Weerawarana} Chairman, CEO, and Founder of WSO2 (an open source application development software company).
